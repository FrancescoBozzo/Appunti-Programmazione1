\chapter{Liste circolari}
Il concetto di \textbf{lista circolare} è del tutto analogo a quello di una lista semplicemente o doppiamente concatenata: l'unica differenza è data dal fatto che l'ultimo elemento della lista è legato al primo. Dunque, bisogna porre attenzione a come scorrere la lista per evitare di incorrere in loop infiniti.

\section{Implementazione della struttura}
Principalmente si possono distinguere due implementazioni per la lista circolare:
\begin{itemize}[noitemsep, nolistsep]
	\item attraverso liste semplicemente o doppiamente concatenate.
	\item attraverso un vettore.
\end{itemize}
\begin{lstlisting}[title={Implementazione di una lista circolare in C++}]
typedef struct ListaCircolare{
    // campi
    Tdato v[MAX];
    unsigned int n;     // numero di elementi nella lista
    unsigned int testa; // indice dell'elemento in testa
    unsigned int coda;  // indice dell'elemento successivo alla coda

    // costruttore
    ListaCircolare(){
        n = testa = coda = 0;
    }
    
    // metodi
    ...
} ListaCircolare;
\end{lstlisting}

\section{Ricerca di un elemento}
\begin{lstlisting}[title={Metodo per la ricerca di un elemento in una lista circolare}]
bool contains(Tdato d){
    int j;
    for (j=0; j<n; j++){
        if (v[(testa+j) % MAX] == d){
            return true;
        }
    }
    return false;
}
\end{lstlisting}

\section{Inserimento di un elemento}
\begin{lstlisting}[title={Metodo per l'inserimento in testa in una lista circolare}]
void insertFirst(Tdato d){
    if (!isFull()){
        testa = (testa-1) % MAX;
        if (testa < 0){
            testa += MAX;
        }
        v[testa] = d;
        n++;
    }
}
\end{lstlisting}
\begin{lstlisting}[title={Metodo per l'inserimento in coda in una lista circolare}]
void insertLast(Tdato d){
    if (!isFull()){
        v[coda] = d;
        coda = (coda+1) % MAX;
        n++;
    }
}
\end{lstlisting}

\section{Rimozione di un elemento}
\begin{lstlisting}[title={Metodo per la rimozione della testa da una lista circolare}]
void removeFirst(){
    if (!isEmpty()){
        testa = (testa+1) % MAX;
        n--;
    }
}
\end{lstlisting}
\begin{lstlisting}[title={Metodo per la rimozione della coda da una lista circolare}]
void removeLast(){
    if (!isEmpty()){
        coda = (coda-1) % MAX;
        if (coda < 0){
            coda += MAX;
        }
        n--;
    }
}
\end{lstlisting}

\section{Calcolare la lunghezza della lista}
Si può notare com'è immediato restituire la lunghezza di una lista circolare:
\begin{lstlisting}[title={Metodo per restituire la lunghezza di una lista circolare}]
int length(){
    return n;
}
\end{lstlisting}

\section{Controllo lista piena/vuota}
In modo analogo, anche qui il campo $n$ (contenente il numero di elementi) torna utile:
\begin{lstlisting}[title={Metodo per controllare se la lista circolare è vuota}]
bool isEmpty(){
    return n == 0;
}
\end{lstlisting}
\begin{lstlisting}[title={Metodo per controllare se la lista circolare è piena}]
bool isFull(){
    return n == MAX;
}
\end{lstlisting}
