\chapter{Introduzione alla programmazione C}

\section{La nascita del C}
La nascita del linguaggio C avviene come conseguenza della necessità di introdurre un nuovo linguaggio di programmazione che renda possibile una maggiore astrazione rispetto al linguaggio macchina; i punti su cui il linguaggio C si basa e che lo hanno reso così importante sono appunto:
\begin{itemize}[noitemsep]
	\item astrazione della memoria: utilizzare alias per identificare determinate celle di memoria.
	\item astrazione delle istruzioni: esprimere istruzioni complesse in un linguaggio comprensibile al programmatore.
	\item astrazione del linguaggio di definizione dell'algoritmo.
\end{itemize}
Il C venne creato nel 1972 da \textit{Brian Kernighan} e \textit{Dennis Ritchie}, ai \textit{Bell Telephone Laboratories} con l’obiettivo di fornire un linguaggio flessibile e vicino alla macchina, tale da diventare il linguaggio di sistema per Unix.\\
Molti linguaggi di programmazione, tra cui il C, necessitano di una macchina astratta per funzionare; questa altro non è che una copia “logica” della macchina di Von Neumann con un ingrediente in più: l’interprete del suo rispettivo linguaggio. Il vantaggio di usare una macchina astratta sta nel fatto che un programma sviluppato in un certo linguaggio può funzionare su qualsiasi hardware su cui è installata la rispettiva macchina astratta.

\section{Astrazione della memoria}
Si definisce come \textbf{variabile} un contenitore di un determinato tipo di dato situato in una specifica porzione di memoria. Il dato contenuto all'interno di una variabile è suscettibile a modifica nel corso dell'esecuzione del programma.
\begin{lstlisting}[title={Dichiarazione di variabili}]
int numero; // dichiarazione di una variabile di tipo intero
char carattere; // dichiarazione di una variabile di tipo carattere
float virgola_1, virgola_2;// dichiarazione di due variabili per contenere valori con la virgola

numero = 1; // inizializzazione della variabile numero
\end{lstlisting}
Una variabile è dunque caratterizzata da un \textit{identificatore} (\textit{left-value}) e da un \textit{valore} (\textit{right-value}). Quando in una espressione utilizziamo l'identificatore di una variabile, durante il calcolo viene considerato il suo valore.\\
Bisogna però prestare attenzione:
\begin{itemize}[noitemsep]
	\item Non è possibile inizializzare o utilizzare una variabile se non è stata precedentemente dichiarata.
	\item Gli identificatori delle variabili possono contenere unicamente caratteri alfanumerici della tabella \textit{ASCII} e "\_". Il primo carattere non può essere un numero.
	\item Gli identificatori delle variabili non possono essere \textit{keyword} del linguaggio C.
	\item Il C è case sensitive: \colorbox{light-gray}{nomeVariabile} è dunque diverso da \colorbox{light-gray}{nomevariabile}.
\end{itemize}

\subsection{L'attributo \textit{const}}
Attraverso l'attributo \textit{const} è possibile dichiarare una costante: essa è del tutto analoga ad una variabile, tranne per il fatto che il suo valore non è modificabile durante il corso del programma.
\begin{lstlisting}[title={Dichiarazione di una costante}]
const float pi = 3.14;
\end{lstlisting}

\section{Le espressioni}
Le \textbf{espressioni} in C vengono valutate seguendo l'albero sintattico indotto dalla \textit{priorità degli operatori} e dalle parentesi.\\
Il valore di un'espressione può essere calcolato anche senza valutare l'intero albero sintattico, ove possibile. Questo meccanismo prende il nome di \textit{lazy-evaluation}.
\begin{lstlisting}
int a = 5;
bool test = (a<4) && (a>0); // in questo caso viene valutata solo (a<4), in quanto il suo valore e' false.
\end{lstlisting}

\section{IO con il terminale}

\subsection{La funzione \textit{printf}}
La funzione \colorbox{light-gray}{printf()} permette di scrivere del testo sulla standard output. Il primo argomento della funzione è una stringa formattata attraverso dei \textit{descrittori di formato} (chiamati anche segnaposto). Seguono tanti argomenti quanti sono i segnaposto della stringa formattata.
In base al tipo di dato che si vuole stampare si utilizzano diversi descrittori di formato:
\begin{itemize}[noitemsep]
	\item \colorbox{light-gray}{\%d} per numeri interi.
	\item \colorbox{light-gray}{\%f} per numeri con virgola.
	\item \colorbox{light-gray}{\%c} per caratteri.
	\item \colorbox{light-gray}{\%s} per stringhe.
\end{itemize}
\`{E} possibile specificare quanti caratteri riservare ad un segnaposto: l'output legato a quel descrittore di formato viene allineato a destra. Se necessario, il padding di default viene effettuato con " ": è comunque possibile specificare un carattere diverso.

\begin{lstlisting}[title={Esempi di output con printf}]
printf("%d + %d = %d", 1, 2, 1+2); 1 + 2 = 3
printf("%2d/%2d/%4d %2d:%2d", giorno, mese, anno, ore, minuti); // padding con " "
printf("%02d/%02d/%4d %02d:%02d", giorno, mese, anno, ore, minuti); // padding con "0"
\end{lstlisting}

All'interno della stringa formattata è possibile utilizzare i caratteri di \textit{escape}:
\begin{itemize}[noitemsep]
	\item \colorbox{light-gray}{"$\backslash$n"} per andare a capo.
	\item \colorbox{light-gray}{"$\backslash$r"} carriage return.
	\item \colorbox{light-gray}{"$\backslash$t"} per una tabulazione.
	\item \colorbox{light-gray}{"$\backslash\backslash$"} per scrivere "$\backslash$".
	\item \colorbox{light-gray}{"$\backslash$""} per scrivere """.
\end{itemize}


\subsection{La funzione \textit{scanf}}
La funzione \colorbox{light-gray}{scanf()} permette di leggere dei dati dallo standard input. La sintassi è del tutto analoga alla funzione di printf, tranne per il fatto di porre una "\&" prima di ogni variabile utilizzata come argomento della funzione (nel caso di array non bisogna utilizzare "\&"). Ulteriori spiegazioni verranno date alle sezioni \ref{puntatori} e \ref{funzioni}.\\
Come di seguito, si consiglia di leggere un solo valore per ciascuna singola chiamata di scanf.
\begin{lstlisting}[title={Esempi di input con scanf}]
int a, b;
char c;
scanf("%d", &a);
scanf("%d", &b);

fflush(stdin); // da utilizzare su Windows prima di leggere un carattere
scanf("%c, &c);
\end{lstlisting}

\begin{lstlisting}[title={Leggere una sequenza di caratteri uno alla volta}]
int c; // getchar restituisce un intero (deve contenere il valore EOF)
while ((c=getchar()) != EOF) { ... }
\end{lstlisting}

\section{Strutture di controllo}

\subsection{Il costrutto \textit{if-else}}
Il costrutto \colorbox{light-gray}{if} permette di compiere delle azioni sulla base della valutazione di una espressione booleana. Il blocco di codice fra parentesi graffe viene infatti eseguito se l'espressione fra parentesi tonde è valutata vera.
\begin{lstlisting}[title={Struttura del costrutto if}]
if (<espressione>){
    ...
}
\end{lstlisting}
Quando si utilizza il costrutto if, è possibile (quindi è opzionale) farlo seguire dal blocco \colorbox{light-gray}{else}.
\`{E} possibile inoltre creare una catena a cascata di \colorbox{light-gray}{else-if}:
\begin{lstlisting}[title={Utilizzo di if, else-if e ramo else}]
if (numero > 0){
    printf("numero positivo");
} else if (numero < 0) {
    printf("numero negativo");
} else {
    printf("numero nullo");
}
\end{lstlisting}

\subsection{Il costrutto \textit{switch}}
In quei casi in cui la catena a cascata di else-if risulta troppo prolissa, è possibile utilizzare il costrutto \colorbox{light-gray}{switch} in maniera del tutto simile:
\begin{lstlisting}[title={Struttura del costrutto switch}]
switch (<integral value>){ // carattere o un numero
    case value1:
        ...
        break;
    case value2:
        ...
        break;
    default:
        ...
        break;
}
\end{lstlisting}
\`{E} importante sottolineare che le istruzioni all'interno del costrutto switch vengono eseguite in sequenza a partire dal \colorbox{light-gray}{case} il cui valore corrisponde a quello della variabile fra parentesi tonde. L'istruzione di \colorbox{light-gray}{break} permette di uscire dal blocco dello switch.

\begin{lstlisting}[title={Significato dell'istruzione break}]
int carattere = 'a';
switch (carattere){
    case 'a':
    case 'e':
    case 'i':
    case 'o':
    case 'u':
        printf("vocale");
    default:
        printf("non e' una vocale");
        break;
}
\end{lstlisting}
Anche se è consigliata, l'istruzione break non è necessaria nel caso di default.\\
Il costrutto switch è molto utilizzato con le enumerazioni.

\section{Cicli}
\subsection{Il costrutto \textit{while}}
Il blocco di codice fra parentesi graffe viene ripetuto fino a quando l'espressione fra parentesi tonde è vera. L'espressione viene valutata prima di eseguire il blocco di istruzioni.
\begin{lstlisting}[title={Struttura del costrutto while}]
while (<espressione>){
    ...
}
\end{lstlisting}

\subsection{Il costrutto \textit{do-while}}
Il blocco di codice fra parentesi graffe viene ripetuto fino a quando l'espressione fra parentesi tonde è vera. L'espressione viene valutata dopo aver eseguito il blocco di istruzioni (dunque il blocco di istruzioni viene eseguito almeno una volta). Spesso è utilizzato nei controlli dell'input.
\begin{lstlisting}[title={Struttura del costrutto do-while}]
do {
    ...
} while (<espressione>);
\end{lstlisting}
\begin{lstlisting}[title={Lettura carattere da tastiera}]
char carattere;
do {
    scanf("%c", &carattere);
} while (carattere<'a' || carattere>'z');
\end{lstlisting}

\subsection{Il costrutto \textit{for}}
Il costrutto \colorbox{light-gray}{for} è una versione specializzata del costrutto while. Si consiglia l'utilizzo del for in quei casi dove si necessita di una variabile contatore.
\begin{lstlisting}[title={Struttura del costrutto for}]
for (<inizializzazione>; <espressione>; <incremento/decremento contatore){
    ...
}
\end{lstlisting}
\`{E} consigliato dichiarare la variabile con funzione di contatore al di fuori del costrutto \colorbox{light-gray}{for}.
\begin{lstlisting}[title={Stampa i numeri da 1 a 10}]
int i;
for (i=1; i<=10; i++){
    printf("%d\n", i);
}
\end{lstlisting}

\subsection{Teorema di Bohem-Jacopini}
\newtheorem*{theorem}{Teorema}
\begin{theorem}[di Bohem-Jacopini]
	Qualunque algoritmo può essere implementato in fase di programmazione utilizzando tre sole strutture dette strutture di controllo: la \textit{sequenza}, la \textit{selezione} ed il \textit{ciclo} (iterazione), da applicare ricorsivamente alla composizione di istruzioni elementari.
\end{theorem}
Come abbiamo visto nelle precedenti sezioni, il C presenta tutte e tre le ipotesi ed è dunque riconosciuto come \textit{linguaggio completo}.

\section{Puntatori}\label{puntatori}
Un puntatore è una variabile che contiene l’indirizzo di un’altra variabile (di un tipo specifico). Esso viene dichiarato utilizzando l’operatore di \textit{dereferenziazione} \colorbox{light-gray}{*}. 
\begin{lstlisting}[title = {Dichiarazione generica di una variabile puntatore}]
tipoDato *nomePuntatore; // puntatore a una variabile di tipo tipoDato
\end{lstlisting}

Un altro operatore utile con i puntatori è \colorbox{light-gray}{\&}, che, se posto a sinistra di una variabile, restituisce l’indirizzo della variabile stessa.
\begin{lstlisting}[title = {Utilizzo dell'operatore \&}]
int numero = 12;
int *ptr = &numero; // ptr contiene l'indirizzo di numero
\end{lstlisting}

\`{E} possibile anche dichiarare puntatori a puntatori:
\begin{lstlisting}[title = {Puntatori di puntatori}]
int numero = 12;
int *singlePtr = &numero; 
int **doublePtr = &singlePtr; // doublePtr contiene l'indirizzo di singlePtr
\end{lstlisting}

Tramite i puntatori si può accedere al right-value di una variabile (puntata) e operare su di esso:
\begin{lstlisting}[title = {Operazioni su variabili tramite puntatori}]
int x = 42;
int *x_pointer;  
// si crea il puntatore a intero "x_pointer"
x_pointer = &x;  
// si associa l'indirizzo di x al puntatore
*x_pointer = 39; 
// tramite la dereferenziazione (*) si opera su x assegnandogli il valore di 39
\end{lstlisting}

\subsection{Aritmetica dei puntatori}
\`{E} possibile eseguire operazioni di incremento, decremento, somma e differenza fra puntatori:
\begin{lstlisting}[title = {Esempi sull'aritmetica dei puntatori}]
int vettore[] = {1, 2, 3};
int *ptr = vettore; // ptr punta alla cella contenente 1
ptr++; // ora ptr punta alla cella di memoria successiva, contenente 2
ptr--; // ora ptr punta alla cella cella contenente 1
ptr = ptr+2; // ora ptr punta alla cella cella contenente 3
int result = ptr - vettore // e' pari a 2, ossia al numero di elementi int fra i due puntatori

\end{lstlisting}

\section{Funzioni}\label{funzioni}
\subsection{Sottoprogrammi: funzioni e procedure}
Le \textbf{funzioni} sono veri e propri sottoprogrammi che una volta inseriti nel codice possono essere riutilizzati più e più volte, senza creare ripetizioni nel codice. Ciò migliora il mantenimento e l'usabilità del codice.\\
In C posso definire due tipi di sottoprogrammi: 
\begin{itemize}
	\item funzioni: restituiscono un valore al chiamante, analogamente alle funzioni matematiche (sia tipi di default che user-defined).
	\item procedure: svolgono un compito per il chiamante ma non restituiscono un valore specifico (restituiscono \colorbox{light-gray}{void}).
\end{itemize}
\underline{Nota}: una funzione non può restituire un array (se necessario restituire un puntatore al primo elemento del vettore).\\
Gli argomenti della funzione, elencati nella testata, vengono chiamati i \textbf{parametri formali} della funzione. I \textbf{parametri effettivi} sono invece quelli con i quali la funzione viene invocata.\\
Per utilizzare una funzione in un programma si deve prima \textit{dichiararla}, ovvero scriverne un prototipo, il prototipo di una funzione viene posto nella parte dichiarativa globale per essere invocata a seguire nel programma. Oltre a questo la funzione deve essere \textit{definita} (ne deve essere descritto il procedimento).  
\begin{lstlisting}[title={Dichiarazione di una funzione}]
// dichiarazione
<return type> functionName (<type arg1> arg1, <type arg2> arg2, ...);

// definizione
<return type> functionName (<type arg1> arg1, <type arg2> arg2, ...){
    ...
}
\end{lstlisting}

\subsection{L'ambiente di una funzione in memoria}
Quando in C un programma richiama una funzione, esso si interrompe momentaneamente per permettere alla macchina di svolgere tale funzione (che è essenzialmente anch’essa un programma): spesso viene creata una specifica macchina dedicata alla funzione con una propria memoria d'ambiente, nettamente separata da quella del programma chiamante.\\
Per gestire gli ambienti delle funzioni chiamate si sfrutta il meccanismo di \textit{gestione della memoria a pila} (\textit{stack}) con una struttura dati \textit{LIFO} (\textit{Last In First Out}).\\ L’invocazione di una funzione comporta l’allocazione di un nuovo ambiente in memoria sulla cima dello stack. Quando termina l’esecuzione del sottoprogramma si recupera il risultato e si rimuove l'ambiente dalla pila.

\subsection{Passaggio dei parametri a funzioni}
In C tutti i parametri passati alle funzioni vengono passati \textbf{per valore}, ossia ne viene effettuata una copia. Dunque, se viene modificato un argomento nel corpo di una funzione, la funzione chiamante non risentirà di questi cambiamenti.
\begin{lstlisting}[title={Il passaggio per valore}]
void cambiaValore(int a){
    a = 6;
}
int main(){
    int a = 5;
    cambiaValore(a);
    printf("%d", a); // stampa 5
}
\end{lstlisting}
Al contrario, utilizzando un puntatore è possibile eseguire un passaggio \textbf{per riferimento}, che consiste nel passare (sempre per valore) il puntatore alla funzione e poi usare il suo  valore per accedere alla variabile a cui punta: se viene modificato in questo modo un argomento nel corpo di una funzione, l'ambiente della funzione chiamante risentirà di questi cambiamenti.
\begin{lstlisting}[title={Il passaggio per riferimento}]
void cambiaValore(int* a){
    *a = 6;
}
int main(){
    int a = 5;
    cambiaValore(&a);
    printf("%d", a); // stampa 6
}
\end{lstlisting}

\`{E} inoltre possibile, attraverso l'attributo \colorbox{light-gray}{const}, impedire la modifica della variabile puntata da puntatore.
\begin{lstlisting}[title={Utilizzo dell'attributo const}]
void cambiaValore(const int* a){
    // non e' possibile modificare a
}
int main(){
    int a = 5;
    cambiaValore(&a);
    printf("%d", a); // stampa 5
}
\end{lstlisting}

Tuttavia gli array sono trattati in modo particolare quando vengono utilizzati come parametro effettivo: viene passato per valore l’indirizzo di base dell’array, cioè l’indirizzo del primo elemento. Inoltre, quando si dichiara una funzione che accetta un vettore come parametro formale, non è necessario indicarne la misura fra parentesi quadre. Al contrario, per matrici di dimensione superiore a 1, è necessario specificare la grandezza di ciascuna dimensione (fatta eccezione per l'ultima) fra parentesi graffe.
\begin{lstlisting}[title={Dichiarazione di funzioni con vettori e matrici come parametri formali}]
void funzione1(int vettore[]);
void funzione2(int matrice[3][]);
void funzione3(int matrice[3][4][2][]);
\end{lstlisting}

\underline{Nota}: In C vige il meccanismo della deroga (\textit{mascheramento}) fra i blocchi di codice: la dichiarazione di un elemento in un blocco maschera eventuali entità omonime esterne al blocco in considerazione. (Un blocco è una sezione di codice delimitata dalle parentesi graffe).

\subsection{Funzioni ricorsive}
Si definisce \textbf{ricorsiva} una funzione che richiama se stessa. Questo meccanismo viene sfruttato per trasformare problemi complessi in problemi elementari di immediata risoluzione.\\
Una funzione ricorsiva è costituita principalmente da due parti:
\begin{itemize}[noitemsep]
	\item \textit{caso base}: soluzione del problema elementare.
	\item \textit{caso induttivo}: divisone del problema complesso in problemi più semplici.
\end{itemize}

\begin{lstlisting}[title={Calcolo del fattoriale di un numero implementato ricorsivamente}]
int fattoriale(int n) {
    if (n < 0){
        return -1;
    }
    if (n == 0){
        return 1;
    } else {
        return n * fattoriale(n-1);
    }
}
\end{lstlisting}
Seppur sia un meccanismo molto potente, programmare per ricorsione non è sempre la scelta più efficiente: ad ogni chiamata di funzione infatti viene creato un nuovo ambiente nello stack. Con input di dimensioni ragionevoli ciò non si nota, ma scalando si possono incontrare facilemente problemi legati all'eccessiva occupazione di memoria.

\subsection{Funzione per la generazione di numeri pseudocasuali}
Con l'ausilio della funzione \colorbox{light-gray}{rand()} inclusa in \colorbox{light-gray}{stdlib.h}, è comodo costruire una funzione che genera numero pseudocasuali compresi fra un minimo e un massimo:
\begin{lstlisting}[title={Funzione per la generazione di numeri pseudocasuali}]
#include <stdlib.h>

int generaNumeroCasuale(int min, int max){
    return (rand() % (max-min+1) + min);
}
\end{lstlisting}

\section{Enumerazioni}
Il costrutto \colorbox{light-gray}{enum} permette di definire un nuovo tipo le cui variabili possono assumere solo un numero finito e predeterminato di valori.
\begin{lstlisting}[title={Utilizzo di un tipo enum}]
enum seme {cuori, picche, fiori, quadri}; // definizione
enum seme semeCarta; // dichiarazione
semeCarta = cuori; // inizializzazione
\end{lstlisting}
In realtà, i membri di un'enumerazione sono rappresentati da costanti intere. Nell'esempio precedente "cuori" è rappresentato da uno 0, "picche" da un 1, e così via. \`{E} inoltre possibile modificare questa numerazione:
\begin{lstlisting}[title={Modifica numerazione di un enum}]
enum seme {cuori=3, picche, fiori, quadri};
// il compilatore automaticamente setta:
//  - picche=4
//  - fiori=5
//  - quadri=6
\end{lstlisting}
Spesso una definizione di un tipo enum è accompagnata da un'operazione di \colorbox{light-gray}{typedef}, che permette di definire un nuovo tipo:
\begin{lstlisting}[title={Utilizzo di un tipo enum con typedef}]
typedef enum seme {cuori, picche, fiori, quadri} seme; // definizione
seme semeCarta; // dichiarazione
semeCarta = cuori; // inizializzazione
\end{lstlisting}
In realtà, attraverso una semplice istruzione del tipo \colorbox{light-gray}{typedef a b;} tutte le occorrenze nel codice di \colorbox{light-gray}{a} vengono sostituite con \colorbox{light-gray}{b}: typedef è infatti semplicemente un operatore sintattico.

\section{Strutture}
\subsection{Definizione e dot notation}
In C è possibile definire una \textbf{struct}, ossia una struttura dati che raggruppa variabili eterogenee (di tipo diverso). Ogni elemento che la compone viene definito \textit{campo} della struttura: data una variabile di tipo struttura è possibile accedere ai campi della stessa utilizzando la \textit{dot notation}.
\begin{lstlisting}[title={Esempio di definizione di struttura}]
// definizione
struct Impiegato{
    char nome[20];
    char cognome[20];
    float stipendio;
    char codiceFiscale[16];
} Impiegato;

struct Impiegato impiegato1; // dichiarazione
impiegato1.stipendio = 1200.50; // dot notation
\end{lstlisting}
Come per le enumerazioni, si può utilizzare anche con le strutture l'operatore sintattico \colorbox{light-gray}{typedef}:
\begin{lstlisting}[title={Esempio di definizione di struttura con typedef}]
typedef struct Punto{
   float x, y;
} Punto;

Punto punto1;
\end{lstlisting}
\`{E} possibile utilizzare anche per le strutture l’operatore unario \colorbox{light-gray}{sizeof}, il quale restituisce il numero di byte occupati in memoria.\\
\underline{Nota}
\begin{itemize}[noitemsep]
	\item non è definito l'operatore \colorbox{light-gray}{==} di confronto fra strutture.
	\item per effettuare la copia di una struttura basta usare l'operatore di assegnazione \colorbox{light-gray}{=}.
\end{itemize}

\subsection{Puntatori a strutture}
Quando vi è la necessità di passare una determinata struttura ad una funzione come parametro, spesso si utilizza il passaggio per riferimento: è infatti molto meno onerosa la copia di un singolo puntatore piuttosto che la copia di una intera struttura. Spesso perciò torna utile il qualificatore \colorbox{light-gray}{const}, che vieta alla funzione di modificare la struttura che le è stata passata per riferimento.\\

Quando si utilizza un puntatore a struttura, per accedere ad un campo della stessa è possibile utilizzare la notazione \colorbox{light-gray}{->} piuttosto che la dot notation.
\begin{lstlisting}[title={Accesso ai campi di un puntatore a struttura}]
Punto p1;
Punto *ptr = &p1;

(*p1).x = 1; // dot notation
p1->y = 2; // notazione ->
\end{lstlisting}
