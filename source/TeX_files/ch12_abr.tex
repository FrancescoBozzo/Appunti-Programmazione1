\chapter{Alberi binari di ricerca}

\section{Definizioni di albero}
\begin{itemize}[noitemsep]
	\item Un \textit{albero} è un grafo non orientato, connesso e privo di cicli.
	\item Attraverso la \textit{ricorsione}:
	\begin{itemize}[noitemsep, nolistsep]
		\item l'insieme vuoto $\phi$ è un albero binario.
		\item $T_s$ e $T_d$ sono alberi binari ed $r$ è un nodo: la terna $(r, T_s, T_d)$ è un albero binario.
	\end{itemize}
	\item Un \textit{albero binario di ricerca} è un albero in cui ogni nodo ha al più due figli tale che ogni nodo è maggiore (o uguale) dei valori dei nodi del suo sottoalbero sinistro e minore (o uguale) dei valori dei nodi del suo sottoalbero destro.
\end{itemize}
Alcune sigle per identificare un albero binario di ricerca sono \textit{ABR} e \textit{BST} (dall'inglese \textit{binary search tree}).

\begin{lstlisting}[title={Implementazione di un albero binario di ricerca}]
typedef struct Tnodo {
Tdato dato;
Tnodo *sx;
Tnodo *dx;
Tnodo *parent; //non utilizzato in lab
} Tnodo;
typedef ABR *Tnodo;
\end{lstlisting}

\section{Proprietà}
\begin{enumerate}[noitemsep]
	\item Un ABR non ha una rappresentazione univoca. Alcuni rappresentazioni infatti risultano più bilanciate di altre.
	\item Un ABR si dice \textit{bilanciato} se i suoi nodi sono equamente distribuiti, senza eccedere in altezza.
	\item \textit{Altezza}: lunghezza massima tra radice e una foglia, ossia $h = n_\text{livelli}-1$.
	\item \textit{Foglia}: nodo senza figli.
	\item \textit{Radice}: nodo senza padre.
	\item \textit{Cammino} di un ABR: somma per ciascun nodo la sua distanza dalla radice.
	\item Il percorso dalla radice ad ogni foglia è unico.
\end{enumerate}


\section{Algoritmi}
\subsection{Visita in preordine}
\begin{lstlisting}[title={Visita in preordine di un ABR}]
void visita_preordine(BST tree){
    if (tree != NULL){
        tree->dato.print();
        visita_preordine(tree->sx);
        visita_preordine(tree->dx);
    }
}
\end{lstlisting}

\subsection{Visita in ordine}
Attraverso una stampa della visita in ordine si ottiene un elenco ordinato.
\begin{lstlisting}[title={Visita in ordine di un ABR}]
void visita_ordine(BST tree){
    if (tree != NULL){
        visita_ordine(tree->sx);
        tree->dato.print();
        visita_ordine(tree->dx);
    }
}
\end{lstlisting}

\subsection{Visita in postordine}
\begin{lstlisting}[title={Visita in postordine di un ABR}]
void visita_postordine(BST tree){
    if (tree != NULL){
        visita_postordine(tree->sx);
        visita_postordine(tree->dx);
        tree->dato.print();
    }
}
\end{lstlisting}

\subsection{Visita in level-ordine}
Nella visita in level-ordine si percorrono i nodi per ciascun livello.

\subsection{Ricerca di un elemento}
In maniera ricorsiva, cerca fra i sottoalberi di un nodo. Nel caso peggiore ha complessità $O(h)$.
\begin{lstlisting}[title={Ricerca ricorsiva di un elemento in un ABR}]
Tnodo *search_element(BST tree, Tdato d){
    if (tree->dato.isEqual(d)){
        return tree;
    } else if (d.isLowerThan(tree->dato)){
        return search_element(tree->sx, d);
    } else {
        return search_element(tree->dx, d);
    }
}
\end{lstlisting}

\subsection{Ricerca del minimo}
Si continua iterativamente a percorrere il sottoalbero sinistro. Nel caso peggiore ha complessità $O(h)$.
\begin{lstlisting}[title={Ricerca del minimo in un ABR}]
Tdato minimo(BST tree){
    BST temp = tree;
    while (temp->sx != NULL){
        temp = temp->sx;
    }
    return temp->dato;
}
\end{lstlisting}

\subsection{Ricerca del massimo}
Si continua iterativamente a percorrere il sottoalbero destro. Nel caso peggiore ha complessità $O(h)$.
\begin{lstlisting}[title={Ricerca del massimo in un ABR}]
Tdato massimo(BST tree){
    BST temp = tree;
    while (temp->dx != NULL){
        temp = temp->dx;
    }
    return temp->dato;
}
\end{lstlisting}

\subsection{Ricerca del successore e predecessore}
Per implementare i seguenti algoritmi, nell'implementazione dell'ABR si necessita del riferimento ai genitori. Nel caso peggiore ha complessità $O(h)$.

\subsection{Calcolo dell'altezza}
L'algoritmo consiste nel percorrere tutti i percorsi, cercandone il più lungo.
\begin{lstlisting}[title={Calcolo dell'altezza in un ABR}]
int altezza(BST tree){
    if (tree == NULL){
       return -1; // dovuto alla definizione di altezza
    } else {
       return max(1+altezza(tree->sx),1+altezza(tree->dx));
    }
}
\end{lstlisting}

\subsection{Inserimento di un elemento}
Nel caso peggiore ha una complessità $O(h)$.
\begin{lstlisting}[title={Inserimento di un elemento in un ABR}]
BST insert_node(BST tree, Tdato d){
    if (tree == NULL){
        return new Tnodo(d, NULL, NULL);
    } else {
        insert_ric(tree, d);
        return tree;
    }
}

// la funzione deve restituire un BST, in quanto non basta creare un nuovo nodo: bisogna anche collegarlo al precedente
BST insert_ric(BST tree, Tdato d){
    if (tree == NULL){
        return new Tnodo(d, NULL, NULL);
    } else {
        if (d.isGreaterThan(tree->dato)){
            tree->dx = insert_ric(tree->dx, d);
        } else {
            tree->sx = insert_ric(tree->sx, d);
        }
    }
    return tree;
}
\end{lstlisting}

\subsection{Conta numero di nodi}
\begin{lstlisting}[title={Conta numero di nodi di un ABR}]
int conta_nodi(BST tree){
    if (tree == NULL){
        return 0;
    } else {
        return 1 + conta_nodi(tree->sx) + conta_nodi(tree->dx);
    }
}
\end{lstlisting}

\subsection{Cancellazione di un elemento}
Di seguito si descrive l'algoritmo per la \textit{cancellazione} di un elemento $z$:
\begin{itemize}
	\item caso 1: se $z$ non ha figlio $\textit{sx}$:
	\begin{enumerate}
		\item se $\text{dx != NULL}$, sostituisci $\textit{dx}$ a $z$.
		\item se ha entrambi i figli $\textit{NULL}$, nessun problema
	\end{enumerate}
	\item caso 2: se $z$ ha solo figlio $\textit{sx}$, sostituisci $\textit{sx}$ a $z$.
	\item caso 3: se $z$ ha due figli. Sia $y$ il successore di $z$ ($y$ ha necessariamente sottoalbero sinistro nullo e si trova nel sottalbero destro di $z$):
	\begin{enumerate}
		\item se $y$ è il diretto figlio destro di $z$: sostituisci $y$ (mantenendo il suo sottoalbero destro) a $z$.
		\item se $y$ non è il diretto figlio destro di $z$: scambia $y$ con il diretto figlio destro di $z$; sostituisci $y$ (mantenendo il suo sottoalbero destro) a $z$.
	\end{enumerate}	
\end{itemize}

