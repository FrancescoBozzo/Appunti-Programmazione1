\chapter{Vettori e matrici}

Il vettore, chiamato più comunemente \textbf{array}, è il più semplice esempio di dato strutturato, esso consiste in una sequenza di celle consecutive ed omogenee, ovvero una sorta di contenitore di tante variabili dello stesso tipo a cui è possibile accedere tramite il nome del vettore e l’indice della variabile racchiuso tra parentesi quadre.\\
Di per se l'array nel linguaggio C non è effettivamente un tipo di dato, esso è un costruttore di tipo che permette di creare un tipo di dato i cui elementi formano una sequenza come descritto poco fa. \\
Un array si crea ponendo l'identificatore del tipo di elemnti che conterrà, il nome dell'array e la dimensione dello stesso tra parentesi quadre nel modo seguente:
\begin{lstlisting}[title={Implementazione di un array}]
//esempio con array con 3 interi
int vettore[3];

//esempio con array di 6 caratteri
char vettore_di_caratteri[6];

//esempio con array di 3 array di 2 interi
int vett[3][2];		//matrice 3*2
\end{lstlisting}
Si possono creare array contenenti qualsiasi tipo di dato, sia esso \textit{built-in} o \textit{user-defined}, semplice o strutturato, come si è appena visto nell'esempio si possono creare anche array di array (detti comunemente array bidimensionali o matrici) e non c'è limte al numero di dimensioni che si possono sfruttare. \\
Tuttavia anche gli array hanno dei difetti, la principale limitazione sta nel fatto che la dimensione di un array non può cambiare durante l'esecuzione del programma; questo implica che nel caso non si conoscano a tempo di programmazione le dimensioni di un input da memorizzare in un array si dovrà necessariamente sovrastimare le dimensioni dell'array per evitare il rischio di \textbf{overflow}, con conseguente spreco memoria.\\ 
Questa complicazione è dovuta alla complessità di realizzazione del compilatore di un linguaggio: per farla breve se la macchina astratta (di un determinato linguaggio) conosce la quantità di memoria necessaria per un programma prima della sua esecuzione 
potrà operare in maniera molto più efficiente riservandosi subito tutta la memoria che le servirà; per questo motivo nella maggior parte dei linguaggi più comuni la possibilità di allocare memoria a tempo di esecuzione è permessa solo in casi speciali.\\
Gli elementi singoli di un array vengono trattati dal C come vere e proprie variabili (del tipo definito nella dichiarazione dell'array) e come queste possono essere coinvolte in tutte le operazioni riguardanti il loro tipo. L'array invece non può essere coinvolto globalmente in operazioni nè di assegnamento nè di confronto. Il metodo più comune per agire sull'array intero spesso prevede l'utilizzo di cicli for in cui l'iteratore fa anche da indice dell'array, come nell'esempio riportato qui sotto.
\begin{lstlisting}[title={Implementazione di un array}]
//si vuole inizializzare il seguente array di interi con una sequenza di 30
int voti[10];
int i;
for(i=0; i<10; i++){
	voti[i]=30;
}
\end{lstlisting}




SCRIVI COSA PUò ESSERE USATO COME INDICE