\chapter{Algoritmi di ricerca e ordinamento}

\section{Algoritmi di ricerca}
Un \textbf{algoritmo di ricerca} permette di trovare un elemento avente determinate caratteristiche all'interno di un insieme di elementi identificati da una chiave.

\subsection{Ricerca sequenziale}
Consiste nel scorrere tutti gli elementi fino a quando non si trova quello desiderato:
\begin{lstlisting}[title={Ricerca sequenziale}]
int sequential_search(char v[], int len, char key){
    int i;
    for (i=0; i<len; i++){
        if (v[i] == key){
            return i;
        }
    }
    return -1;
}
\end{lstlisting}
In questo caso, la funzione $T(n)$ è pari:
\begin{itemize}[noitemsep, nolistsep]
	\item nel caso peggiore: $n$.
	\item nel caso migliore $1$.
	\item nel caso medio $\frac{n}{2}$.
\end{itemize}

\subsection{Ricerca sequenziale con sentinella}
Come sopra, ma evita un confronto: posiziono al termine del vettore l'elemento da cercare:
\begin{lstlisting}[title={Ricerca sequenziale con sentinella}]
// la funzione chiamante deve assicurare la presenza di un elemento vuoto alla fine del vettore
int sequential_search_sentinella(char v[], int len, char key){
    int i;
    v[len] = key;
    for (i=0; v[i]!=key; i++){	}

    if (i == len) {
        return -1;
    }
    return i;
}
\end{lstlisting}

\subsection{Ricerca sequenziale su vettore ordinato}
Questa ricerca sfrutta il fatto di ricevere un vettore ordinato in input:
\begin{lstlisting}[title={Ricerca sequenziale su vettore ordinato}]
int sequential_search_order(char v[], int len, char key){
    int i;
    for (i=0; key>=v[i] && i<len; i++){
        if (v[i] == key){
            return i;
        }
    }
    return -1;
}
\end{lstlisting}

\subsection{Ricerca binaria}
Si presuppone che l'input sia un insieme ordinato di elementi con la possibilità di accesso casuale (\textit{es.} array):
\begin{lstlisting}[title={Ricerca binaria}]
int binary_search(char v[], int len, char key){
    int bottom = 0;
    int top = len - 1;
    int mid;

    while (bottom <= top){
        mid = (top + bottom) / 2;
        if (v[mid] == key) {
            return mid;
        }
        if (key < v[mid]){
            top = mid - 1;
        } else {
            bottom = mid + 1;
        }
    }
    return -1;
}
\end{lstlisting}
In questo caso, la funzione $T(n)$ è pari:
\begin{itemize}[noitemsep, nolistsep]
	\item nel caso peggiore: $\log_2n$.
	\item nel caso migliore $1$.
	\item nel caso medio $\log_2n$.
\end{itemize}

\section{Algoritmi di ordinamento}

\subsection{Insertion sort}
L'algoritmo di \textbf{insertion sort} permette di riordinare in maniera efficiente un piccolo insieme di elementi.
La complessità asintotica dell'algoritmo:
\begin{itemize}[noitemsep, nolistsep]
	\item caso medio: $\varTheta(n^2)$
	\item caso peggiore: $\varTheta(n^2)$
\end{itemize}
Dato un vettore $v$, l'algoritmo si basa sul costruisce (a partire dall'indice zero di $v$) un sottoarray ordinato.
\begin{lstlisting}[title={Insertion sort}]
void insertion_sort(char v[], int len){
    int i, j;
    char temp;
    for (i=1; i<len; i++){
        j = i-1;
        temp = v[i];
        while (j>=0 && v[j]>temp){
            v[j+1] = v[j];
            j--;
        }
        v[j+1] = temp;
    }
}
\end{lstlisting}

\subsection{Merge Sort}
Viene qui di seguito presentata la versione ricorsiva dell'algoritmo di ordinamento \textbf{merge sort}. Esso è definito come un algoritmo \textit{divide et impera}, in quanto:
\begin{itemize}[noitemsep]
	\item[divide]: separa la sequenza di $n$ elementi in 2 da $\frac{n}{2}$.
	\item[impera]: ordina le due sottosequenze generate da divide.
	\item[combina]: unisce le due sottosequenze rispettivamente ordinate in un tempo $\varTheta(n)$.
\end{itemize}
Utilizzando l'implementazione ricorsiva, sono indispensabili i casi base:
\begin{itemize}[noitemsep, nolistsep]
	\item dato un vettore da un elemento, esso è gia ordinato.
	\item dato un vettore di due elementi, o è già ordinato o è necessario invertirne gli elementi.
\end{itemize}
La complessità asintotica dell'algoritmo:
\begin{itemize}[noitemsep, nolistsep]
	\item caso medio: $\varTheta(n\log n)$
	\item caso peggiore: $\varTheta(n\log n)$
\end{itemize}
\begin{lstlisting}[title={Merge Sort}]
// passo di combina, con allocazione dinamica della memoria
void merge(char v[], int p, int mid, int r){
    int i, j, k;
    int len1 = mid-p+1;
    int len2 = r-mid;
    char *L = new char[len1+1];
    char *R = new char[len2+1];

    for (i=0; i<len1; i++){
        L[i] = v[p+i];
    }
    for (i=0; i<len2; i++){
        R[i] = v[mid+1+i];
    }
    L[len1] = R[len2] = (char)'z'; // sentinella

    i = j = 0;
    for (k=p; k<=r; k++){
        if (L[i] < R[j]){
            v[k] = L[i++];
        } else {
            v[k] = R[j++];
        }
    }
    delete [] L;
    delete [] R;
}

void merge_sort(char v[], int p, int r){
    // caso base: 1 elemento
    if (r-p <= 0){
        return;
    // caso base: 2 elementi
    } else if (r-p == 1) {
        if (v[p] > v[r]){
            char temp = v[p];
            v[p] = v[r];
            v[r] = temp;
        }
        return;
    // caso induttivo
    } else {
        int mid = (p+r)/2;
        merge_sort(v, p, mid);
        merge_sort(v, mid+1, r);
        merge(v, p, mid, r);
    }
}

\end{lstlisting}

\subsection{Teorema della complessità nel caso pessimo di un algoritmo di ordinamento}
La complessità nel caso pessimo di un algoritmo di ordinamento sul posto che confronta e scambia elementi consecutivi è $\varOmega(n^2)$. Algoritmi più efficienti richiedono scambi di elementi lontani fra loro.