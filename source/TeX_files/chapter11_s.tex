\chapter{Gestione file}
\section{Il concetto di file}
Il C gestisce le interazioni con le varie periferiche fisiche rappresentandole come \textbf{file} su cui agisce attraverso un meccanismo chiamato stream, che funziona come un'interfaccia consistente, che cioè permette di interagire con tutti i tipi di periferiche allo stesso modo. Un file viene associato ad uno \textbf{stream} (flusso) tramite un operazione di open, che da avvio alle comunicazioni tra software e periferica.\\ Mentre tute le periferiche non lavorano alla stessa maniera i file sì, per questo è comodo intermediare le comunicazioni tramite i file così da renderle standard; il lingaggio C ha almeno 3 file costantemente aperti: stdin (file di input), stdout (file di output) e stderr (file di registrazione degli errori).\\
\section{Come implementare un file}
Nella libreria standard del C esistono comandi diversi per operare su file di testo e file binari, la funzione di apertura del flusso è però uguale in entrambi i casi e si chiama \textbf{fopen()}, dove all'interno delle parentesi va scritto ("nomedelfile.estensione", "metododiapertura"); i possibili metodi di apertura sono:
\begin{itemize}
	\item \textbf{"w"}: \textit{write} apre il flusso in modalità scrittura su file di testo
	\item \textbf{"r"}: \textit{read} apre il flusso in modalità lettura da file di testo
	\item \textbf{"a"}: \textit{append}  apre il flusso in modalità "aggiunta" a file di testo\footnote[1]{La modalità aggiunta è necessaria poichè ogni volta che un file viene aperto con comando "w" o "wb" il suo contenuto precedente viene eliminato}
	\item \textbf{"wb"}: comando "w" per file binari
	\item \textbf{"rb"}: comando "r" per file binari
	\item \textbf{"ab"}: comando "a" per file binari
\end{itemize}
Esistono anche comandi avanzati ("w+", "r+", "a+") che permettono di modificare il cursore del file e quindi decidere in che posizione agire.\\
La funzione \textbf{fopen()} inoltre restituisce l'indirizzo della struttura file associata al flusso che crea,mentre restituisce NULL se c'è stato qualche errore in fase di apertura del flusso.
\section{Operazioni di lettura e scrittura su file}
In questa sezione vengono presentati i metodi di lettura e scrittura su file, con esempi di file binari e di testo.
\begin{lstlisting}[title={File di testo (.txt)}]
//creazione del puntatore alla struttura file
FILE *f_txt;
int x=7;
char testo[20];
//apertura file in modalita' scrittura con controllo sull'apertura
if((f_txt=fopen("nomefile.txt", "w"))==NULL){
	printf("errorre apertura file fase w");
}
//scrittura sul file
fprintf(f_txt, "numero_piccioni: %d"/*(tipo della variabile da scrivere)*/, x);
fprintf(f_txt, "si puo' scrivere anche solo testo", NULL);
//chiusura del file (e dello stream)
fclose(f_txt);

//apertura file in modalita' lettura 
if((f_txt=fopen("nomefile.txt", "r"))==NULL){
printf("errorre apertura file fase r");
}
//lettura
fscanf(f_txt, "%s: %d"/*tipi scritti esattamente come sono scritti nel file*/, &testo, &x);
//chiusura stream
fclose(f_txt);

//apertura file modalita' append
if((f_txt=fopen("nomefile.txt", "a"))==NULL){
printf("errorre apertura file fase a");
}
//append
fprintf(f_txt, "testo aggiunto\n", NULL);
//chiusura
fclose(f_txt);
\end{lstlisting}
\begin{lstlisting}[title={File binari (.bin)}]



\end{lstlisting}