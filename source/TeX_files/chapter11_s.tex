\chapter{Gestione file}
\section{Il concetto di file}
Il C gestisce le interazioni con le varie periferiche fisiche rappresentandole come file su cui agisce attraverso un meccanismo chiamato stream, che funziona come un'interfaccia consistente, che cioè permette di interagire con tutti i tipi di periferiche allo stesso modo. Un file viene associato ad uno stream tramite un operazione di open, che da avvio alle comunicazioni tra software e periferica.\\ Mentre tute le periferiche non lavorano alla stessa maniera i file sì, per questo è comodo intermediare le comunicazioni tramite i file, il lingaggio C ha almeno 3 file costantemente aperti: stdin, stdout e strderr.\\
\section{Come implementare un file}

