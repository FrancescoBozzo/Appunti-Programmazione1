\chapter{Gestione file}
\section{Il concetto di file}
Il C gestisce le interazioni con le varie periferiche fisiche rappresentandole come \textbf{file} su cui agisce attraverso un meccanismo chiamato stream. Esso funziona come un'interfaccia consistente, cioè permette di interagire con tutti i tipi di periferiche allo stesso modo. Un file viene associato ad uno \textbf{stream} (flusso) tramite un'operazione di \textit{open}, (avvio delle comunicazioni tra software e periferica).\\
Il lingaggio C ha almeno 3 file costantemente aperti: \colorbox{light-gray}{stdin} (file di input), \colorbox{light-gray}{stdout} (file di output) e \colorbox{light-gray}{stderr} (file di registrazione degli errori).

\section{Come apriree chiudere un file}
Nella libreria standard del C esistono diversi strumenti per operare su file di testo e file binari. In entrambi i casi la funzione di apertura del flusso è però la medesima:
\begin{lstlisting}[title={Interfaccia funzione open()}]
FILE *file = fopen(<nome del file>, <tipo di apertura>);
\end{lstlisting}
La funzione \colorbox{light-gray}{fopen()} inoltre restituisce l'indirizzo della struttura file associata al flusso creato; se al contrario incontra qualche errore, restituisce \textit{NULL}.
Di seguito sono elencate le diverse modalità di apertura disponibili:
\begin{itemize}[noitemsep]
	\item \colorbox{light-gray}{"w"}: scrittura su file di testo.
	\item \colorbox{light-gray}{"r"}: lettura da file di testo.
	\item \colorbox{light-gray}{"a"}: aggiunta a file di testo.\footnote[1]{La modalità aggiunta è necessaria poichè ogni volta che un file viene aperto con comando "w" o "wb" il suo contenuto precedente viene eliminato}
	\item \colorbox{light-gray}{"wb"}: scrittura su file binari.
	\item \colorbox{light-gray}{"rb"}: lettura da file binari.
	\item \colorbox{light-gray}{"ab"}: aggiunta a file binari.
\end{itemize}
Esistono anche modalità avanzate (\colorbox{light-gray}{"w+"}, \colorbox{light-gray}{"r+"}, \colorbox{light-gray}{"a+"}) che permettono di modificare il cursore del file, decidendo in che posizione del file agire.\\
Ad ogni apertura di un file deve corrispondere una chiusura, attraverso la funzione \colorbox{light-gray}{fclose()}:
\begin{lstlisting}[title={Interfaccia funzione close()}]
fclose(<file>);
\end{lstlisting}
\section{Operazioni di lettura e scrittura su file}
Per la scrittura su un file di testo:
\begin{lstlisting}[title={Scrittura su file di testo}]
fputc(int c, file_output); // ritorna EOF come errore
fputs(stringa, file_output); // ritorna EOF come errore
fprintf(file_output, [formato_output], [var_list variabili]); // ritorna il numero di elementi scritti
\end{lstlisting}

Per la lettura da un file di testo:
\begin{lstlisting}[title={Lettura da file di testo}]
carattere = fgetc(file_input); // ritorna EOF come errore
fgets(stringa, lunghezza, file_input); // fino a newline o EOF; ritorna EOF come errore
fscanf(file_input, [formato_input], [var_list puntatori]); // ritorna il numero di elementi letti
\end{lstlisting}

Per verificare se ci si trova alla fine del file:
\begin{lstlisting}[title={Controllo fine file}]
int result = feof(file_input); // se la fine e' stata raggiunta, result assume un valore diverso da 0
\end{lstlisting}
Seppur la libreria standard del C offre questa funzione, quando bisogna leggere in insieme di elementi di cui non si conosce la dimensione è consigliato utilizzare \colorbox{light-gray}{fscanf()}. In particolare il controllo da eseguire è sul valore di ritorno di fscanf: se non sono stati letti tanti elementi quanti ci si aspetterebbe, allora il file è concluso.\\

Per lettura e scrittura su file binario:
\begin{lstlisting}[title={Controllo fine file}]
int n_read_elem = fread (void *ptr, len_elemento, n_elementi, file_input);
int n_written_elem = fwrite(void *ptr, len_elemento, n_elementi, file_input);
\end{lstlisting}
Una caratteristica interessante delle funzioni relative ai file binari è quella di offrire la possibilità di lettura/scrittura di tipi di dato, sia \textit{built-in} che \textit{user-defined}, con una singola istruzione.

\section{Alcuni esempi}
\begin{lstlisting}[title={Scrittura su file di testo}]
// creazione del puntatore alla struttura file
FILE *f_txt;
int x = 7;

// apertura file in modalita' scrittura con controllo sull'apertura
if ((f_txt=fopen("nomefile.txt", "w")) == NULL){
	printf("errore apertura file fase w");
}

// scrittura sul file
fprintf(f_txt, "numero_piccioni: %d", x);
fprintf(f_txt, "si puo' scrivere anche solo testo", NULL);

fclose(f_txt);
\end{lstlisting}

\begin{lstlisting}[title={Lettura da file di testo}]
FILE *f_txt;
if((f_txt=fopen("nomefile.txt", "r"))==NULL){
    printf("errore apertura file fase r");
}

int x;
char testo[20];
fscanf(f_txt, "%s: %d", &testo, &x); //lettura

fclose(f_txt);
\end{lstlisting}

\begin{lstlisting}[title={Append di un file di testo}]
FILE *f_txt;
if ((f_txt=fopen("nomefile.txt", "a")) == NULL){
    printf("errore apertura file fase a");
}

fprintf(f_txt, "testo aggiunto\n", NULL); // append

fclose(f_txt);
\end{lstlisting}

\begin{lstlisting}[title={File binari (.bin)}]
//creazione del puntatore alla struttura file
FILE *f_bin;
int x=12;

if ((f_txt=fopen("nomefile.bin", "wb")) == NULL){ 
    printf("errore apertura file fase wb");
}

// se si sta programmando su Windows, per poter controllare il file si deve comunque dichiararlo come .txt

fwrite(&x, sizeof(int), 1, f_bin); // scrittura sul file bin
fclose(f_bin);
\end{lstlisting}

\begin{lstlisting}[title={Scrittura di una struttura su file binario}]
typedef struct Tpunto {
    float x, y;
} Tpunto;

Tpunto p1;
p1.x = p2.x = 0.3;
FILE *f_bin;
if ((f_txt=fopen("nomefile.bin", "wb")) == NULL){ 
    printf("errore apertura file fase wb");
}

fwrite(&p1, sizeof(Tpunto), 1, f_bin); // scrittura binaria di un elemento Tpunto
fclose(f_bin);
\end{lstlisting}

\begin{lstlisting}[title={Lettura di una struttura da file binario}]
typedef struct Tpunto {
    float x, y;
} Tpunto;

Tpunto p1;
FILE *f_bin;
if((f_txt=fopen("nomefile.bin", "rb")) == NULL){
    printf("errore apertura file fase rb");
}

fread(&p1, sizeof(Tpunto), 1, f_bin); // lettura binaria di un elemento Tpunto

fclose(f_bin);
\end{lstlisting}

\begin{lstlisting}[title={Scrittura di una stuttura su file binario in append}]
typedef struct Tpunto {
    float x, y;
} Tpunto;

Tpunto p1;
p1.x = p2.x = 0.3;
FILE *f_bin;
if ((f_txt=fopen("nomefile.txt", "ab")) == NULL){
    printf("errore apertura file fase ab");
}

fwrite(&d2, sizeof(Tdato), 1, f_bin); // scrittura binaria di un elemento Tpunto

fclose(f_bin);
\end{lstlisting}