\chapter{Stack e code}

\section{Stack}
\subsection{Definzione e implementazione}
Lo \textbf{stack} (in italiano \textbf{pila}) è una struttura dati astratta di tipo \textbf{LIFO} (\textit{Last In First Out}).\\
Sia per definizione, che per problemi di ottimizzazione, lo stack prevede un solo punto di accesso per l'inserimento e la rimozione di un elemento: la testa.\\
La pila può essere implementata con:
\begin{itemize}[noitemsep, nolistsep]
	\item \textit{array}: gestione statica della memoria; limite massimo di dimensione.
	\item \textit{liste}: gestione dinamica della memoria; occupazione variabile e non limitata.
\end{itemize}

\subsection{Funzioni}
Le principali funzioni che uno stack deve fornire sono:
\begin{itemize}[noitemsep]
	\item \textit{init()}: inizializza la struttura.
	\item \textit{push()}: inserisce in elemento in testa con complessità $O(1)$.
	\item \textit{pop()}: elimina un elemento dalla testa con complessità $O(1)$.
	\item \textit{top()}: restituisce l'elemento in testa.
	\item \textit{isEmpty()}: verifica se lo stack è vuoto.
	\item \textit{isFull()}: verifica se lo stack è pieno.
	\item \textit{remove()}: rimozione di un elemento (utilizzando un secondo stack).
\end{itemize}

\subsection{Applicazioni}
Alcuni utilizzi dello stack riguardano:
\begin{itemize}[noitemsep]
	\item trasformazione da espressioni in notazione infissa a postfissa.
	\item valutazione di espressioni in notazione postfissa.
	\item implementazione iterativa di algoritmi riguardo gli alberi binari di ricerca.
	\item struttura degli ambienti di chiamata delle funzioni in runtime.
\end{itemize}

\section{Code}
\subsection{Definizione}
La \textbf{coda} (in inglese \textbf{queue}) è una struttura dati astratta di tipo \textbf{FIFO} (\textit{First In First Out}).\\
Sia per definizione, che per ottimizzare sia l'estrazione che l'inserimento, la queue prevede 2 punti di accesso:
\begin{itemize}[noitemsep, nolistsep]
	\item \textit{front} (in italiano \textit{testa}) per l'estrazione.
	\item \textit{rear} (in italiano \textit{coda}) per l'inserimento.
\end{itemize}

\subsection{Implementazione}
La coda può essere implementata attraverso:
\begin{itemize}[noitemsep]
	\item liste concatenate.
	\item liste circolari.
\end{itemize}
\begin{lstlisting}[title={Implementazione di una coda attraverso liste concatenate}]
typedef struct Queue {
    Tnodo *head;
    Tnodo *tail;
} Queue;
\end{lstlisting}

\subsection{Funzioni}
Le principali funzioni che una queue deve fornire sono:
\begin{itemize}[noitemsep]
	\item \textit{enqueue()}: inserisce un elemento nel rear con complessità $O(1)$.
	\item \textit{dequeue()}: estrae un elemento dal front con complessità $O(1)$.
	\item \textit{queueIsEmpty()}: verifica se la coda è vuota.
	\item \textit{queueIsFull()}: verifica se la coda è piena (se implementata con liste circolari).
	\item \textit{front()}: restituisce l'elemento in front.
\end{itemize}