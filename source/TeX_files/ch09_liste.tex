\chapter{Liste}
\section{Implementazione della struttura}
Una \textbf{lista} è una struttura di dati astratta che permette di contenere un insieme di elementi. Attraverso la gestione dinamica della memoria non è inoltre necessario specificare la dimensione della struttura al momento della compilazione (attraverso costanti o \textit{define}). A differenza di un array, in cui i propri elementi occupano celle contigue di memoria, gli elementi di una lista possono essere localizzati posizioni diverse dell'heap.\\
Una lista può essere implementata in due diverse modalità:
\begin{itemize}
	\item \textbf{semplicemente concatenata (sc)}: ogni elemento ha un puntatore all'elemento successivo
	\item \textbf{doppiamente concatenata (dc)}: ogni elemento ha due puntatori: uno all'elemento precedente e l'altro all'elemento successivo
\end{itemize}

Seppur la presenza di un ulteriore puntatore nell'implementazione \textit{dc} possa sembrare un'inutile ridondanza, in realtà permette di semplificare e ottimizzare alcuni algoritmi.\\
Entrambe le strutture presentano un metodo di accesso \textit{sequenziale} e non casuale (tipico degli array).

\begin{lstlisting}[title={Implementazione lista singolarmente concatenata}]
typedef struct Tnodo{
    Tdato dato;
    Tnodo *next;

    // costruttori
    Tnodo(Tdato d);
    Tnodo(Tdato d, Tnodo *n);
} Tnodo;
typedef Tnodo *ListaSemplicementeConcatenata;
\end{lstlisting}

\begin{lstlisting}[title={Implementazione lista doppiamente concatenata}]
struct Tnodo {
    Tdato dato;
    Tnodo *next;
    Tnodo *prev;
    
    // costruttori
    Tnodo(Tdato);
    Tnodo(Tdato d, Tnodo *p, Tnodo *n);
};
typedef Tnodo *ListaDoppiamenteConcatenata;
\end{lstlisting}

\section{Ricerca di un elemento}
Come anticipato precedentemente, si può accedere ad un elemento di una lista in maniera sequenziale e non casuale. Dunque, nel caso peggiore, la complessità dell'algoritmo è $O(n)$.
\begin{lstlisting}[title={Ricerca di un elemento di una lista semplicemente/doppiamente concatenata}]
Lista ricerca(const Lista ls, Tdato d){
    Lista temp = ls;
    while (!is_empty_lista(temp)){
        if (temp->dato.isEqual(d)) {
            return temp;
        }
        temp = temp->next;
    }
    return NULL;
}
\end{lstlisting}


	
\section{Inserimento di un elemento}
Nel caso di una lista \textit{sc}, quando la si scorre per inserire un elemento in coda o in ordine, è necessario tenere memoria anche dell'elemento precedente a quello in cui si sta visitando.\\
\`{E} molto oneroso effuare operazioni in coda: è necessario infatti visitare tutta la lista.
\begin{lstlisting} [title={Inserimento in testa, in coda, ordinato in una lista \textit{sc}}]
Lista inserisci_in_testa(Lista ls, Tdato d){
    return new Tnodo(d, ls);
}

Lista inserisci_in_coda(Lista ls, Tdato d){
    Lista temp = ls;
    if (is_empty_lista(temp)){
        return inserisci_testa(ls, d);
    }
    while (temp->next != NULL){
        temp = temp->next;
    }
    temp->next = new Tnodo(d, NULL);
    return ls;
}

Lista inserisci_ordinato(Lista ls, Tdato d){
    Lista temp = ls;
    Lista pred = NULL;
    if (is_empty_lista(temp)){
        return inserisci_testa(ls, d);
    }
    while ((!is_empty_lista(temp)) && temp->dato.isLowerThan(d)){ 
        //isLowerThan=metodo per verificare se dato e' minore di d
        pred = temp;
        temp = temp->next;
    }
    if (is_empty_lista(temp)){
        pred->next = new Tnodo(d, NULL);
    } else if (temp==ls) {
        return inserisci_testa(ls, d);
    } else {
        pred->next = new Tnodo(d, temp);
    }
    return ls;
}
\end{lstlisting}
\vspace{20px}
Invece, nel caso di una lista doppiamente concatenata, è possibile risalire all'elemento precedente attraverso al campo \textit{prev}:
\begin{lstlisting}[title={Inserimento in testa, in coda, ordinato in una lista \textit{dc}}]
ListaDoppia inserisci_testa_doppia(ListaDoppia ls,Tdato d){
    ListaDoppia temp = new Tnodo(d, NULL, ls);
    if (ls != NULL){
        ls->prev = temp;
    }
    return temp;
}

ListaDoppia inserisci_coda_doppia(ListaDoppia ls, Tdato d){
    if (ls == NULL){
        return inserisci_testa_doppia(ls, d);
    }
    ListaDoppia temp = ls;
    while (temp->next != NULL){
        temp = temp->next;
    }
    temp->next = new Tnodo(d, temp, NULL);
    return ls;
}

ListaDoppia inserisci_ord_doppia(ListaDoppia ls, Tdato d){
    if (ls == NULL){
        return inserisci_testa_doppia(ls, d);
    }
    ListaDoppia temp = ls;
    ListaDoppia old = NULL;
    while ((temp!=NULL) && (temp->dato.isLowerThan(d))){
        old = temp;
        temp = temp->next;
    }
    Tnodo *t = NULL;
    if (temp == NULL){
        t = new Tnodo(d, old, NULL);
        old->next = t;
    } else {
        if (temp->prev != NULL){
            t = new Tnodo(d, temp->prev, temp);
            temp->prev->next = t;
            temp->prev = t;
        } else {
            t = new Tnodo(d, NULL, temp);
            temp->prev = t;
            return t;
        }	
    }	
    return ls;
}
\end{lstlisting}

\section{Rimozione di un elemento}
Come per l'inserimento, è molto oneroso effuare operazioni in coda: è necessario infatti visitare tutta la lista.
\begin{lstlisting}[title={Eliminazione di un elemento, di un elemento in testa e in coda con lista \textit{sc}}]
Lista delete_el(Lista ls, Tdato d){
    Lista temp = ls;
    Lista pred = NULL;
    while (!is_empty_lista(temp)){
    if (temp->dato.isEqual(d)){
        if (pred != NULL){
            pred->next = temp->next;
            delete temp;
            return ls;
        } else {
            Lista res = temp->next;
            delete temp;
            return res;
        }
    }
    pred = temp;
    temp = temp->next;
    }
}

Lista delete_testa(Lista ls){
    if (ls != NULL) {
        Lista temp = ls;
        ls = ls->next;
        delete temp;
    }
    return ls;
}

Lista delete_coda(Lista ls){
    if (ls != NULL) {
        Lista temp = ls;
        Lista pred = NULL;
        while (!is_empty_lista(temp->next)){
            pred = temp;
            temp = temp->next;
        }
        if (pred == NULL){
            return delete_testa(ls);
        }
        delete temp;
        pred->next = NULL;
    }
    return ls;
}
\end{lstlisting}

\begin{lstlisting}[title={Eliminazione di un elemento, di un elemento in testa e in coda con lista \textit{dc}}]
ListaDoppia delete_el_doppia(ListaDoppia ls, Tdato d){
    if (ls != NULL){
        if (ls->dato.isEqual(d)){
            return delete_testa_doppia(ls);
        }
        ListaDoppia temp = ls;
        while ((temp!=NULL) && (!temp->dato.isEqual(d))){
            temp = temp->next;
        }
        if (temp != NULL){
            if (temp->next != NULL){
                temp->next->prev = temp->prev;
            }
            temp->prev->next = temp->next;
            delete temp;
        }
    }
    return ls;
}

ListaDoppia delete_testa_doppia(ListaDoppia ls){
    ListaDoppia temp = ls->next;
    delete ls;
    temp->prev = NULL;
    return temp;
}

ListaDoppia delete_coda_doppia(ListaDoppia ls){
    if (ls != NULL){
        ListaDoppia temp = ls;
        while (temp->next != NULL){
            temp = temp->next;
        }
        if(temp->prev != NULL){
            temp->prev->next = NULL;
            delete temp;
        } else {
            delete temp;
            return init_doppia();
        }
    }
    return ls;
}
\end{lstlisting}

\section{Calcolare la lunghezza di una lista}
Sia per le liste \textit{sc} e \textit{dc} l'implementazione è la medesima:
\begin{lstlisting}
unsigned int length_lista(const Lista ls){
    unsigned int len = 0;
    Lista temp = ls;
    while (!is_empty_lista(temp)){
        len++;
        temp = temp->next;
    }
    return len;
}
\end{lstlisting}
